
\documentclass[12pt]{article}

\usepackage{macros}
\makeglossaries

\newacronym{lmm}{LMM}{Libor Market Model}
\newacronym{libor}{Libor}{London Interbank Offered Rate}
\newacronym{ibor}{Ibor}{Interbank Offered Rate}
\newacronym{euribor}{Euribor}{European Interbank Offered Rate}
\newacronym{ls}{Longstaff-Schwartz}{Longstaff Schwartz method}
\newacronym{usd}{USD}{US Dollar}


\newglossaryentry{cra}
{
	name=Callable Range Accrual,
	description=Callable Range Accruals est \dots
}
\newglossaryentry{ra}
{
	name= Range Accrual,
	description={},
}
\newglossaryentry{ks}
{
	name={Kolodko-Schoenmakers},
	description={}
}

\newglossaryentry{future}
{
	name={future},
	description={}
}
\newglossaryentry{option}
{
	name={option},
	description={}
}
\newglossaryentry{swap}
{
	name={swap},
	description={}
}
\newglossaryentry{mc}
{
	name={Monte-Carlo},
	description={}
}
\newglossaryentry{cap}
{
	name={cap},
	description={}
}
\newglossaryentry{caplet}
{
	name={caplet},
	description={}
}
\newglossaryentry{swaption}
{
	name={swaption},
	description={}
}
\newglossaryentry{spot-libor}
{
	name={spot-Libor},
	description={}
}
\newglossaryentry{tenor}
{
	name={Tenor Structure},
	description={}
}
\newglossaryentry{swapment}
{
	name={swapment},
	description={}
}
\newglossaryentry{hjm}
{
	name={Heath-Jarrow-Morton},
	description={}
}

\title{
	\huge Pricing of the Callable Range Accruals using Longstaff Schwartz algorithm
}

\author{
	\begin{tabular}[t]{c@{\extracolsep{4em}}c@{\extracolsep{4em}}c@{\extracolsep{4em}}c}
		Houda Sebti${}^1$ & Said Guida${}^2$\\
	\end{tabular}
	{}\\
	\\
	${}^1$       Ecole des Ponts ParisTech, Champs-sur-Marne, France\\
	${}^2$       Efficiency Management Consulting, Paris, France
	{}\\
	\\
	sebti.houda14@gmail.com\\
}
\begin{document}
	
	\maketitle
	\thispagestyle{empty}
	\pagestyle{empty}
	
	\begin{abstract}
		
		The Callable Range Accrual is one of the most widely used derivatives. The aim of this paper is to model this product under the \acrlong{lmm}.
		
		The first step of the work is the mathematical modeling of the \gls{cra}. To do this, we  used the method of \gls{ls} to determine the sequence of optimal stopping times.\\
		
		The second step of the work consists of using both stock price and rates as the underlyings of \gls{cra}. To do so, we will use Black Scoles model for stock prices and \acrshort{lmm} model for LIBOR rates. We will then calibrate the \acrshort{lmm} to the market data and then implementing the C ++ program allowing the pricing of the model. To do so, we used the financial open source library Quantlib.\\
		
		In the last step of this work, we will adress the application of a corrective measure to the model in order to reduce the gap between the market price and the model price and to correct the relative pricing error due to the calibration.
	\end{abstract}
	
	\section{INTRODUCTION}
	
		blablabla
	

		\subsection{Black Scholes formula}
		The Blac
		\subsection{libor rate}
		
		blablabla
		\subsection{Libor Market Model}
		
			blablaba
		\subsection{Callable Range Accruals}
			\begin{itemize}
				\item {\small \bf Range accruals:}\\
				A range accrual, or range accrual note, is a type of financial derivative product where one party pays a fixed coupon\footnote{rate of interest that a bondholder receives} to a second party and receives a variable coupon from it.\\
				The variable coupon depends on the value of an index. This index could be an interest rate, currency exchange rate, the price of a commodity, or stock index. If the index value falls within a specified range, the coupon accrues or is credited interest. If the index value falls outside the specified range, the coupon rate does not accumulate.
				\\
				\item {\small \bf The callable aspect:}\\
				A callable (or cancellable) instrument gives the exotic interest rate payer (variable leg payer) the option to terminate the contract prior to its maturity, and on predetermined dates.
			\end{itemize}
			\section{Cash flows of a callable range accrual}
				
				\begin{figure}[H]
					\begin{center}
						\caption{\label{fig::cash_flow_cra}\acrlong{cra} cash flows}
						\begin{scaletikzpicturetowidth}{\textwidth}
							\begin{tikzpicture}[scale=\tikzscale]
							\tikzstyle{lien}=[>=stealth, rounded corners = 5pt, thick]
							%%
							\draw[->][ultra thick](-0.5,0)--(8, 0);
							\draw[ultra thick][color=purple](-0.5, -0.2)--(-0.5, 0.2);
							\draw[above left][color=purple!20](2, 0.3) node[draw, circle, fill=purple!20]{\scriptsize{\textcolor{black}{${T}_1$}}};
							\draw[->][thick][color=gray](1.9, 0)--(1.9, 2);
							\draw[above left](1.9, 2) node{\small{FC 1 payment}};
							
							\draw[above right][color=purple!20](3, 0.3) node[draw, circle, fill=purple!20]{\scriptsize{\textcolor{black}{${T}_2$}}};
							\draw[->][thick][color=gray](3, 0)--(3, 2);
							\draw[above right](2.6, 2) node{\small{FC 2 payment}};
							
							\draw[above right][color=purple!20](4.5, 0.3) node[draw, circle, fill=purple!20]{\scriptsize{\textcolor{black}{${T}_3$}}};
							\draw[->][thick][color=gray](4.9, 0)--(4.9, 2);
							\draw[above right](4.5, 2) node{\small{FC 3 payment}};
							
							
							\draw[->][thick][color=gray](7.5, 0)--(7.5, 3);
							\draw[above right][color=purple!20](13.9, 0.3) node[draw, circle, fill=purple!20]{\scriptsize{\textcolor{black}{${T^{"}}_n$}}};
							\draw[above right](13.9, 3) node{\small{fixed coupon n paymen}};
							
							\draw[->][thick][color=gray](2.9, 0)--(2.9, -3);
							
							\draw[above left][color=purple!60](2.9, 0) node[draw, circle, fill=purple!60]{\scriptsize{\textcolor{white}{$T_1$}}};
							\draw[above](2.9, -3) node{\small{$1^{er}$ accrual coupon}};
							
							\draw[->][thick][color=gray](6.9, 0)--(6.9, -2.5);
							
							\draw[above left][color=purple!60](6.9, 0) node[draw, circle, fill=purple!60]{\scriptsize{\textcolor{white}{$T_2$}}};
							\draw[above](6.9, -2.5) node{\small{$2^{eme}$ accrual coupon}};
							
							\draw[->][thick][color=gray](15.9, 0)--(15.9, -3);
							
							\draw[thick][color=purple](3.4, -1.2)--(3.4, -1.6);
							
							\end{tikzpicture}
						\end{scaletikzpicturetowidth}
						
					\end{center}
				\end{figure}
			
				The cash flows $Z_{i}$ at time $T_i$ of a callable range accrual can be modeled as follows:
				\begin{equation}
					Z_i=\frac{\delta_{T_i} . C_i -\delta_{T_i}^{'} . F_i}{B(T_i)}
				\end{equation}
				$C_{i}$ is the variable leg coupon paid at date $T_i$.\\
				$F_{i}$ are the fixed leg coupons.\\
				$B(T_{i)}$ is the actualization factor, it's equal to $e^{-rT_{i}}$.\\
				$[T_{1}, T_{2},..., T_{n}]$ are the fixed tenor dates: the dates in which we pay the fixed leg coupons ${F_{i}}_{i \in [|1,n|]}$.\\
				$[{T_{1}}^{'}, {T_{2}}^{'},..., {T_{n}}^{'},]$ are the variable tenor dates: the dates in which we pay the variable leg coupons ${C_{i}}_{i \in [|1,n|]}$.\\
				${\delta}_{i} = T_{i+1} - T_{i},  \forall i \in [|1,n|]$\\
				${{\delta}_{i}}^{'} = {T_{i+1}}^{'} - {T_{i}}^{'},  \forall i \in [|1,n|]$\\

				\subsection{Fixed leg coupons $F_{i}$}
				\subsection{Variable leg coupons $C_{i}$}
					\begin{equation*}
						C_i=payoff * \frac{\sum_{j=i - 1}^{j=i}\indicator (S_{t_i} \in [S_{min},S_{max}])}{T_i - T_{i-1}} * \delta(T_{i-1}, T_i)
					\end{equation*}
					where $\delta(T_1, T_2) $ is the day-count fraction from date $T_1$ to date $T_2$:\\
					$$
					\delta(T_1, T_2) = \frac{T_2 - T_1}{365}
					$$
					when we replace $\delta(T_{i-1}; T_i)$ by its expression, this equation becomes:\\
					$\begin{array}{lcl}
					C_i=payoff * \frac{\sum_{j=i - 1}^{j=i}\indicator (S_{t_i} \in [S_{min},S_{max}])}{T_i - T_{i-1}} * \frac{T_i - T_{i-1}}{365} \\

					=payoff * \frac{\sum_{j=i - 1}^{j=i}\indicator (S_{t_i} \in [S_{min},S_{max}])}{365}
					\end{array}$
					\begin{equation*}
					\E(C_i)=payoff * \frac{\sum_{j=i}^{j=i+1} \PP (S_{t_i} \in [S_{min},S_{max}])}{365}
					\end{equation*}
					
					\begin{equation}
						\PP(S_t \in [S_{min},S_{max}]) = \int_{S_{min}}^{S_{max}} \frac{1}{\sqrt{2 \pi}} \frac{1}{s\sigma \sqrt{t}} \exp(-\frac{[\ln(S)-\ln(S_0)-
							(\mu-\frac{\sigma^2}{2})t]^2}{2 \sigma ^{2}t}) ds
					\end{equation}
					
					Let:  $$u = ln(s)$$\\
					So:  $$du = \frac{ds}{s}$$
					The equation becomes:
					\begin{equation}
					\PP(S_t \in [S_{min},S_{max}]) = \int_{ln(S_{min})}^{ln(S_{max})} \frac{1}{\sqrt{2 \pi}} \frac{e^{-u}}{\sigma \sqrt{t}} \exp(-\frac{[u-\ln(S_0)-
						(\mu-\frac{\sigma^2}{2})t]^2}{2 \sigma ^{2}t}) e^{u}du
					\end{equation}
					Equation (3) implies:
					\begin{equation}
					\PP(S_t \in [S_{min},S_{max}]) = \int_{ln(S_{min})}^{ln(S_{max})} \frac{1}{\sqrt{2 \pi}} \frac{1}{\sigma \sqrt{t}} \exp(-\frac{[u-\ln(S_0)-
						(\mu-\frac{\sigma^2}{2})t]^2}{2 \sigma ^{2}t}) du
					\end{equation}
					\begin{equation}
					\PP(S_t \in [S_{min},S_{max}]) = F_{Y}(S_{max}]) - F_{Y}(S_{min}])
					\end{equation}
					\\
					where $F_{X}(x)$ is the distribution function of a Gaussian random variable X
					\begin{equation}
					F_{X}(x) = \frac{1}{2}[1 + erf(\frac{x - \mu}{\sigma \sqrt{(2)}})])
					\end{equation}
					with:
					 $$X \sim N(\mu,\sigma)$$
					{\it erf} is the error function given by:
					\begin{equation}
					erf(x) = \frac{2}{\sqrt{\pi}} \int_{0}^{x} e^{-t^{2}} dt
					\end{equation}
					in our case,
					\begin{equation}
					 Y~N[ln(S_{0}) + (\mu - \frac{\sigma^{2}}{2}t) ; \sigma^{2}t])
 					\end{equation}
 					$\begin{array}{lcl}
 						\PP(S_t \in [S_{min},S_{max}]) =\\ \frac{1}{2}[1+erf(\frac{ln(S_{max})-ln(S_0)-(\mu-\frac{\sigma^2}{2}t)}{\sigma \sqrt{(2t)}})]-\frac{1}{2}[1+erf(\frac{ln(S_{min})-ln(S_0)-(\mu-\frac{\sigma^2}{2}t)}{\sigma \sqrt{(2t)}})]
 					\end{array}$
 				
 					$\begin{array}{lcl}
 					\PP(S_t \in [S_{min},S_{max}]) =\\ \frac{1}{2}[erf(\frac{ln(S_{max})-ln(S_0)-(\mu-\frac{\sigma^2}{2}t)}{\sigma \sqrt{(2t)}})]-erf(\frac{ln(S_{min})-ln(S_0)-(\mu-\frac{\sigma^2}{2}t)}{\sigma \sqrt{(2t)}})]
 					\end{array}$
	 	\subsection{Exercise value}
	 	
		 	Is the value of the contract at a certain date\\
		 	$\begin{array}{lcl}
			 	E(T_i) = \sum_{k = 1}^{i} (C_k * e^{-r(T_i - T_k)})\\
			 	=payoff * \sum_{k=1}^{i}\frac{\sum_{j=i - 1}^{j=i}\indicator (S_{t_j} \in [S_{min},S_{max}])}{365} * e^{-r(T_i - T_k)}
		 	\end{array}$
		 	
		 	
	 	\subsection{Hold value}
	 	The hold value at a date $T_i$ is the expected future value of the contract if it has not been canceled at $T_i$, knowing the filtration of information we have today.
	 	$\begin{array}{lcl}
		 	H(T_i) = \E(E(T_{i}) * e^{r(\tau-T_i)} + \sum_{t \in[T_{i+1}; \tau ]}E(t)*e^{-r(t - T_i)}|S_{T_i})
	 	\end{array}$
	 	
	 	$$
		\left\{
		 	\begin{array}{ll}
		 	\tau & \mbox{ is the stopping time if the contract is cancelled before its maturity T, and after $T_i$} \\
		 	\tau & \mbox{ is the maturity T if the contract is held after $T_i$  and until it matures}
		 	\end{array}
	\right.
	 	$$
	\section{CALLABLE RANGE ACCRUAL PRICING}
	\subsection{State of the art}
	blabla
	\subsection{Longstaff Schwartz algorithm}
	blabla
	\subsection{computing the conditional expectancy}
	blabla
	
	\section{Experiments}
	
	\subsection{Data}
	
	blablabla
	
	\subsection{Results}]
	blabla
	\section{CONCLUSIONS}
	
	A conclusion section is not required. Although a conclusion may review the main points of the paper, do not replicate the abstract as the conclusion. A conclusion might elaborate on the importance of the work or suggest applications and extensions. 
	
	\addtolength{\textheight}{-12cm}  
	
	
	\section*{APPENDIX}
	
	Appendixes should appear before the acknowledgment.
	
	\section*{ACKNOWLEDGMENT}
	
	The preferred spelling of the word ÒacknowledgmentÓ in America is without an ÒeÓ after the ÒgÓ. Avoid the stilted expression, ÒOne of us (R. B. G.) thanks . . .Ó  Instead, try ÒR. B. G. thanksÓ. Put sponsor acknowledgments in the unnumbered footnote on the first page.
	
	
	
	References are important to the reader; therefore, each citation must be complete and correct. If at all possible, references should be commonly available publications.
	
	
	
	\begin{thebibliography}{99}
		
		\bibitem{c1} G. O. Young, ÒSynthetic structure of industrial plastics (Book style with paper title and editor),Ó 	in Plastics, 2nd ed. vol. 3, J.
		
	\end{thebibliography}
	
	
	
	
\end{document}
